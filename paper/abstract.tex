This paper presents the Frontier Tree exploration algorithm, a novel approach to autonomously explore unknown and unstructured areas. Focus of this work is the exploration of domestic environments with arbitrary obstacles, for example furbished appartements. Existing and well studied approaches like greedy algorithms perform worse, when obstacles are included and the range of distance sensors is limited. Base of this research is the Frontier Tree. This data structure offers two main features. It is a memory of past poses during exploration and is utilized to decide between future navigation goals. This approach is compared to a basic but efficient nearest neighbour exploration. The algorithm is tested in simulation with maps similar to appartement ground maps including furniture as obstacles. The paper shows, that frontier trees minimize the travelled path of the mobile robot by tracking open boundaries.
